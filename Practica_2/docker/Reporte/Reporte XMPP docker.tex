\documentclass[10pt,letterpaper]{article}
\usepackage[utf8]{inputenc}

\usepackage[spanish]{babel}
\usepackage{hyperref}
\usepackage{graphicx}
\graphicspath{ {/home/jsebastian-ar/Documentos/Latex_images/Prosody/} }
\usepackage[export]{adjustbox}

\begin{document}
\title{Montar Servidor XMPP y BD de postgresql en Contenedor Docker}
\author{Acosta Rosales Jair Sebastián}
\date{\today}
\maketitle

\section{Intoducción XMPP}
XMPP, cuyo significado es Extensible Messaging and Presence Protocol, es un protocolo de mensajeria basado en XML, el cual permite un servicio de mensajerio entre usuarios que permite 

\section{Descargando imagen Docker para Postgresql 11.4}

Para este apartado no se creará un archivo Dockerfile para postgresql, en lugar de ello se implementará una imagen oficial desde la página \textbf{hub.docker.com} en donde es posible encontrar imagenes de múltiples servicios, uno de ellos de bases de datos, como lo pueden ser el propio postgrsql, mysql, etc.

Por lo tanto, al acceder a la página hub.docker.com y buscar postgresql, deberá aparecer lo siguiente:\\


\begin{figure}[htb]
\centering
\includegraphics[scale=.3]{postgresql1}
\end{figure}

El cual es la imagen oficial para postgresql, haciendo click a este enlace nos dirigira al sitio con todas las imagenes posibles para descargar e implementar en docker, iremos al apartado de la derecha donde se encuentra un comando de docker, en la parte de abajo habrá un enlace con lo siguiente \textbf{View Available Tags} lo que significa que ahí podremos ver las versiones disponibles de postgresql, y descargar la que deseemos.

\begin{center}
\includegraphics[scale=.5]{postgresql2}
\end{center}

Para este ejemplo se busco y se selecciono la versión 11.4 de postgresql, en la siguiente imagen podemos observar el comando con el cual podemos descargar esta imagen, ubicado en la esquina superior derecha, el cual es: \textbf{docker pull postgres:11.4}\\

\begin{center}
\centering
\includegraphics[scale=.3]{postgresql3}
\end{center}

Ejecutando el comando en consola esperaremos a que la imagen sea descargada en nuestra computadora y poder crear un contenedor con este.

\begin{center}
\includegraphics[scale=.4]{postgresql4}\\
En esta imagen se hizo el pull con la versión 11.7 de postgresql
\end{center}

Una vez descargado la nueva imagen procedemos a revisar las imagenes disponibles en nuestra version local de docker, con el comando \textbf{docker images}, donde se desplegará la lista de todas las imágenes disponibles.

\begin{center}
\includegraphics[scale=.5]{postgresql5}
\end{center}

\clearpage
\section{Creación de imagen Docker para servidor XMPP}

El servidor XMPP debe ser 

\end{document}